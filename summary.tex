\documentclass[12pt]{amsart}
\usepackage[text={7in,10in}, centering]{geometry}                % See geometry.pdf to learn the layout options. There are lots.
\geometry{letterpaper}                   % ... or a4paper or a5paper or ... 
%\geometry{landscape}                % Activate for for rotated page geometry
%\usepackage[parfill]{parskip}    % Activate to begin paragraphs with an empty line rather than an indent
\usepackage{graphicx}
\usepackage{amssymb}
\usepackage{epstopdf}
\DeclareGraphicsRule{.tif}{png}{.png}{`convert #1 `dirname #1`/`basename #1 .tif`.png}

\title{CS 109 Course Summary}
\author{Daniel Jackoway}
%\date{}                                           % Activate to display a given date or no date

\begin{document}
\maketitle

This document has formulas and other useful information about the first half of CS 109. The idea is that it should basically summarize everything we've learned so far and be super-valuable for the midterm.

\section{Random Variables}
%
\subsection{Binomial}
\subsubsection{What}
\[
X \sim \text{Bin}(n,p)
\]
Do n independent trials with probability of success p. X is the number of successes.
\subsubsection{Formulae}
\[
PMF: P(X = i) = p(i) = {n \choose i} p^i (1-p)^{n-i}
\]
\[
\text{CMF: sum the PMF from 0 to i}
\]
\subsubsection{Combining}
If they have the \textbf{same p}, you add their n's when you add them.
%
\subsection{Negative Binomial}
%
\subsection{Normal}
%
\subsection{Exponential}
%
\subsection{Geometric}
%
\subsection{Hypergeometric}
%
\subsection{Poisson}


\end{document}
